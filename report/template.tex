%%%%%%%%%%%%%%%%%%%%%%%%%%%%%%%%%%%%%%%%%%%%%%%%%%%%%%%%%%%%%%%%%%%%%%%%%%%%%%%%
%2345678901234567890123456789012345678901234567890123456789012345678901234567890
%        1         2         3         4         5         6         7         8

\documentclass[letterpaper, 11 pt, conference]{ieeeconf}  % Comment this line out if you need a4paper


\usepackage{hyperref}
\usepackage{url}
\usepackage[pdftex]{graphicx}
\usepackage{amsfonts}
\usepackage{subfigure} 
\usepackage{algorithm}
%\usepackage{algorithmic}
\usepackage{amsmath}
\usepackage{algcompatible}
\usepackage{framed}
\usepackage{balance}

\pdfminorversion=4

\IEEEoverridecommandlockouts                              % This command is only needed if 
                                                          % you want to use the \thanks command
\overrideIEEEmargins                                      % Needed to meet printer requirements.


\title{\LARGE \bf
RL Transfer Learning
}


\author{Mahmudur Rahman, Sums Uz Zaman}


\begin{document}

\maketitle
\thispagestyle{empty}
\pagestyle{empty}


\begin{abstract}

adsasd adasdasd  adsasd a dasd asd asd asdasdas dasdas das asdasdasd adsasd adasdasd  adsasd a dasd asd asd asdasdas dasdas das asdasdasd adsasd adasdasd  adsasd a dasd asd asd asdasdas dasdas das asdasdasd adsasd adasdasd  adsasd a dasd asd asd asdasdas dasdas das asdasdasd adsasd adasdasd  adsasd a dasd asd asd asdasdas dasdas das asdasdasd adsasd adasdasd  adsasd a dasd asd asd asdasdas dasdas das asdasdasd adsasd adasdasd  adsasd a dasd asd asd asdasdas dasdas das asdasdasd adsasd adasdasd  adsasd a dasd asd asd asdasdas dasdas das asdasdasd

\end{abstract}


\section{Introduction}

Humans and some other animales are good at varieties of tasks and can learn a new task very easily. The main catalyst of human as well as other animal's fast learning capability is  their use of of previous knowledge to learn a new task. They can incorporate all the relevent information in the context of the new task from all of their previous experiences. Transferring skills from previously learned tasks to the new tasks or in the new environments is also important in the context of Reinforcement Learning (RL) agents. If a Reinforcement Learning (RL) agent can use the previously learned skills to a new task efficiently, it would learn the new task faster than learning from the scratch. The main challange is to enable the agent to use the previous experiences efficiently with the change of the task and the type of the environment. In this project, we used a common feature embedding method to transfer skills from the source task to the target task in order to learn the target task faster.

\bibliographystyle{abbrv}
%\bibliography{YourBibFile}
\end{document}



\end{document}
